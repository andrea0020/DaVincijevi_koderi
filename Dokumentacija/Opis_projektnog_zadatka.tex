\chapter{Opis projektnog zadatka}
		
		\textbf{\textit{dio 1. revizije}}\\
		 Cilj ovog projekta je razviti programsku podršku za stvaranje web aplikacije  ” Što želiš čitati?” kojoj je glavni zadatak omogućiti korisniku pronalazak ponuditelja knjige u blizini koji ima u prodaji traženu knjigu na hrvatskom ili srodnom jeziku. Te omogućuje i međusobno povezivanje ponuditelja knjiga radi proširenja međusobnih ponuda. Naime ljubitelji čitanja koji žele čitati stranu literaturu imaju problem s pronalaskom željenih knjiga na hrvatskom zato što mnogi naslovi nisu prevedeni na hrvatski ili srodni jezik poput srpskog i bosanskog, a u slučaju da postoji prijevod zbog loše ažurnosti web stranica ponuditelja knjiga nije ih lako pronaći.  

        Stoga je ideja ove web aplikacije da omogući čitatelju unos značajke željene knjige, imena i ili ponuditelja knjige te pretraga dostupnosti po navedenim značajkama.  Potencijalna korist ovog projekta je što olakšava i ubrzava čitateljima pronalazak tražene knjige, a u slučaju da one ne postoji ne oduzima im puno vremena, te tako poboljšava njihovo čitalačko iskustvo.  Dodatna potencijalna korist je što čitatelju omogućava proširivanje horizonata i upoznavanje sa stranom kulturom čitajući stranu literaturu. Rezultat toga bi bila promocija kulture čitanja na području Republike Hrvatske. S druge strane, ponuditelji knjiga, poput izdavača, antikvarijata i preprodavača, imaju priliku proširiti svoj doseg na širu publiku te povećavajući prodaju knjiga iz svoje ponude. To pomoglo malim poduzetnicima koji drže antikvarijate i nezavisne knjižare da ostanu u poslu, a osnažilo bi izdavače  da prošire svoju ponudu i prevedu još i više stranih naslova, te bi samim time i kupci bili zadovoljniji. 

        U aplikaciji postoje tri uloge:
        \begin{packed_item}
			\item \textit{Neregistrirani korisnik: potraživač knjiga, kupac}
			\item \textit{Registrirani korisnik: ponuditelj knjiga}
			\item \textit{Administrator }
		\end{packed_item}
        Neregistrirani korisnik ima mogućnost pristupu web sjedištu aplikacije, te može pretraživati ponude knjiga. Pretraga se provodi po značajkama knjige koju koje neregistrirani korisnik unosi i po nazivu ponuditelja. Neregistriranom korisniku se nudi pregled karte na kojoj su označene lokacije (adrese) svih ponuditelja knjiga te ih on može izabrati. Nakon odabira ponuditelja neregistriranom korisniku prikazuje se popis, to jest ponuda svih knjiga izabranog ponuditelja. Ponuđene knjige sadržavaju sljedeće značajke:
        \begin{packed_item}
			\item \textit{naziv,}
			\item \textit{autore,}
			\item \textit{godinu izdanja,}
			\item \textit{izdavača,}
			\item \textit{kategorija izdavača (domaći, strani),}
			\item \textit{žanr,}
            \item \textit{ISBN,}
			\item \textit{broj izdanja,}
			\item \textit{stanje očuvanosti,}
			\item \textit{tekstni opis,}
			\item \textit{sliku korica,}
			\item \textit{oznaku vrste knjige,}
            \item \textit{listu ponuda.}
		\end{packed_item}
        S time da oznaku vrste knjige možemo okarakterizirati u 5 različitih kategorija:
        \begin{packed_item}
			\item \textit{Knjiga je na stranom jeziku (npr. engleski), a ne postoji izdanje na hrvatskom ili srodnom jeziku (bosanski, srpski) }
			\item \textit{Knjiga je izdana na hrvatskom jeziku i dobavljiva je na području Hrvatske}
			\item \textit{Knjiga je izdana na hrvatskom jeziku, ali nije dobavljiva na području Hrvatske (npr. izdavač je rasprodao izdanje)}
			\item \textit{Knjiga je izdana na srodnom jeziku, dobavljiva je samo na njihovom tržištu }
			\item \textit{Knjiga je izdana na srodnom jeziku, dobavljiva je u Hrvatskoj, ne postoji na hrvatskom jeziku}
		\end{packed_item}
        Dok lista ponude svake knjige treba uključivati:
        \begin{packed_item}
			\item \textit{naziv ponuditelja,}
			\item \textit{broj dostupnih primjeraka,}
			\item \textit{cijenu knjige kod dotičnog ponuditelja.}
		\end{packed_item}
        Uz to da ako knjiga nema više dostupnih registriranih  ponuditelja knjigu se ne prikazuje u ponudi prilikom pretrage u sustavu.
        Osim odabira ponuditelja i pretrage ponuđenih knjiga, neregistriranom korisniku se daje i mogućnost da kroz sučelje aplikacije zatraži od registriranog korisnika izdavača da stavi u ponudu prijevod željene knjige i samim time u dogovoru sa stranim izdavačem prevede knjigu na hrvatski jezik. Zahtjevi za prijevod se jednostavno bilježe na stranici i akumuliraju se za svaku knjigu na stranom jeziku kako bi ponuditelj imao uvid o potražnji prijevoda. Važno je još napomenuti da ostala komunikacija vezana uz nabavu knjige između neregistriranog korisnika i ponuditelja se ne odvaja kroz aplikaciju već preko drugih kanala poput e-pošte i telefonskim pozivom. 
        
        Registrirani korisnik je isključivo ponuditelj knjiga koji spada u jednu od tri kategorije:
        \begin{packed_item}
			\item \textit{Izdavač}
			\item \textit{Antikvarijat}
			\item \textit{Preprodavač}
		\end{packed_item}
        Ponuditelj bira svoju kategoriju tijekom registracije, pri čemu registraciju treba odobriti administrator nakon pregleda zahtjeva. Nadalje registrirani korisnik unosi potrebne informacije za kontakt ponuditelja, a to su: 
        \begin{packed_item}
			\item \textit{naziv,}
			\item \textit{e-pošta,}
			\item \textit{adresa,}
			\item \textit{izdavača,}
			\item \textit{broj telefona.}
		\end{packed_item}
        Nakon toga registrirani korisnik pristupa web aplikaciji uz pomoć korisničkog imena i lozinke. Svaki ponuditelj ima mogućnost ponuditi neograničeni broj naslova knjiga i njezinih primjeraka u skladu sa svojom kategorijom. Tijekom kreiranja ponude svaki ponuditelj je primoran ispravno odrediti oznaku vrste knjige, te pri svakom dodavanju nove knjige ili primjeraka već postojeće knjige u web aplikaciju. Uz to pri svakoj izmjeni ponude knjiga poput dodavanja novih primjeraka već postojeće knjige ili uklanjanju primjeraka koji su prodani ponuditelj je dužan provjeriti da oznaka vrste knjige odgovara stvarnosti i ako to nije slučaj, izmijeniti ju. Ponuditelj također nije obavezan ponuditi sve knjige koje ima u ponudi, to jest u svojoj ponudi može imati više knjiga nego što ih je ponudio na web stranici.
        Izdavač u svojoj ponudi smije nuditi knjige isključivo knjige na hrvatskom jeziku, te na temelju prikupljenih zahtjeva neregistriranih korisnika smije zatražiti izdavača strane knjige za dozvolu prijevoda knjige sa stranog ili srodnog jezika na hrvatski jezik. 
        Antikvarijat u svojoj ponudi smije imati knjige na stranom jeziku, srodnom jeziku ili hrvatskom jeziku, s time da se mora svojom adresom nalaziti isključivo na području Hrvatske.
        Preprodavač u svojoj ponudi može imati sve vrste knjiga neovisno o prijevodu, te i one koje nisu na drugačiji način dobavljive na području Hrvatske. Uz to adresa mu može biti u Hrvatskoj i u zemljama sa srodnim jezikom, odnosno Srbiji i Bosni i Hercegovini.
       
        Skup korisnika koji bi mogao biti zainteresiran za ostvareno rješenje se može raščlaniti na četiri skupa:
        \begin{packed_item}
			\item \textit{Ljubitelji knjiga: Oni koji traže stranu literaturu prevedene na hrvatski jezik ili srodne jezike.}
			\item \textit{Izdavači: Oni koji žele proširiti dostupnost svojih knjiga na novo tržište.}
			\item \textit{Antikvarijati: Oni koji imaju rijetke ili stare knjige koje žele ponuditi zainteresiranim čitateljima i kolekcionarima.}
		\end{packed_item}
        Aplikacija bi se u budućnosti mogla proširiti na različite načine koji bi poboljšali zadovoljstvo neregistriranih i registriranih korisnika:
        U slučaju da u ponudi nema tražene knjige čitatelju bi se moglo ponuditi knjiga u ponudi od istog autora ili slične tematike, te na taj način čitatelj bi dobio štivo za čitanje, a ponuditelj bio ostvario dodatnu prodaju. 
        Mogla bi se omogući komunikacija kupca i ponuditelja unutar aplikacije kako bi se kupcu olakšala kupnja.
        Moglo bi se omogućiti neregistriranim korisnicima da uz zahtjev za prijevodom određenog naslova mogu unijeti email kako bi mogli biti obaviješteni o izlasku prijevoda i proširenju ponude.
