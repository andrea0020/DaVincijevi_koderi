\chapter{Zaključak i budući rad}
		

		
		 Naša grupa za projektni zadatak je dobila izradu i razvoj web aplikacije pod nazivom: „Što želiš čitati?“  koja omogućuje čitateljima pretragu ponuditelja knjiga na mapi, te pretragu knjiga na temelju njihove dostupnosti na hrvatskom jeziku i tržištu, također im i omogućavam zahtjev za prijevod knjige ako on već ne postoji. S druge strane ponuditeljima knjiga, izdavačima, antikvarijatima i preprodavačima omogućava ponudu knjiga većoj publici i tako poboljšanje poslovanja. Kroz četrnaest tjedana timskog rada uspješno smo ostvarili cilj i uz to smo stekli nova znanja i vještine. Projekt se razvijao u dvije faze. \\
		
		 Prva faza je započela okupljanjem tima, dijeljenjem ideja o razvoju projekta, određivanju ciljeva, te izražavanju pojedinačnih interesa i kompetencija na području razvoja web aplikacije, te razvoja samog projekta. Uzevši sve u obzir raspodijelili smo potrebni posao po mogućnostima i željama svakoga člana, te smo odredili rokove predaje. Podijelili smo se u dva tima, prvi od četiri člana zadužen za dokumentaciju i drugi od tri člana zaduženi za početnu implementaciju. Prva faza je trajala do kolokvija uz malo kašnjenje pri predaji. \\
   
          Nakon što smo dobili ocjenu prve faze i pomoćne savjete što popraviti i kako se poboljšati u drugoj fazi, sastali smo se, raspravili smo o tome što je dobro bilo u prvoj fazi, a što je pošlo po zlu i uzrokovalo kašnjenje, te što možemo učiniti u drugoj fazi da to izbjegnemo. Zaključili smo da bi nam češći sastanci, i raspodjela zadataka na manje komponente s češćim rokovima predaje pomogla da uspješno i na vrijeme odradimo vlastite zadatke. I to bi nam omogućilo praćenje ako tko ima kakvih problema sa svojim zadatkom i brzu intervenciju, kako bi izbjegli veće zaostatke u razvoju.  Budući da smo se svi planirali baviti implementacijom u drugoj fazi, tim za implementaciju iz prve faze nam je predstavio što su oni bili implementirali i predložio kako da se raspodijelimo. Međusobno smo raspodijelili pravljenje potrebne dokumentacije, te smo ju napravili na početku druge faze kako bi se mogli nesmetano prebaciti na implementaciju.  S time smo bili uglavnom uspješni uz nedostatak par funkcionalnosti poput provjere adresa registriranih korisnika. To jest ne ograničavamo registrirane korisnike na područje djelovanja njihove vrste kako je specijalizirano u zadatku. Također omogućili smo da sve knjige imaju mogućnost zahtjeva za prijevod, a ne samo one na stranom jeziku.  Te kod registracije ne provjeravamo je li željena e-mail adresa ili ime već zauzeto. \\
          
          Sve u svemu sudjelovanje u ovom projektu je vrijedno iskustvo za svakoga člana, osim što smo naučili kako funkcionira izrada web aplikacije, kako se koristiti JavaScript bibliotekom React, okvir za razvoj Java aplikacija Spring, LaTeX, Git i GitHub te ostale tehnologije, naučili smo kako je to raditi u timu.  Spoznali smo važnost dobre komunikacije i kako dobro postavljeni rokovi i dobro definirani zadatci olakšavaju izvođenje projekta. Svjesni smo da ima prostora za poboljšanje, ali zadovoljni smo postignutim uspjehom i znanjima koje smo stekli, te se veselimo budućem radu i potencijalnoj suradnji.
		
		\eject 
