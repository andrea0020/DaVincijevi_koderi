\documentclass[12pt]{article}
\usepackage{amsmath}
\usepackage{graphicx}
\usepackage{hyperref}
\usepackage[latin1]{inputenc}

\title{Korištene tehnologije i alati}

\begin{document}
\maketitle

Komunikacija u timu odvijala se uživo te preko aplikacija Whatsapp i Discord.

UML dijagrami izrađeni su u aplikaciji Astah\footnote{https://astah.net}, koja izdaje besplatnu studentsku licencu, te pomoću stranica VisualParadigm\footnote{https://online.visual-paradigm.com} i draw.io\footnote{https://app.diagrams.net}. Dijagram baze podataka izrađen je u online alatu ERDPlus\footnote{https://erdplus.com}.
Git\footnote{https://github.com} je služio kao sustav za upravljanje izvornim kodom, a udaljeni repozitorij projekta dostupan je na GitHubu.

Razvojno okruženje bio je Microsoft Visual Studio\footnote{https://visualstudio.microsoft.com}, a korišten je i uređivač koda Visual Studio Code\footnote{https://code.visualstudio.com}, Microsoftov program pogodan za izradu web aplikacija, koji, iako nije IDE, proširenjima (extensions) nudi podršku za mnoge programske jezike i alate kao što su debuggeri i intellisense.

Baza podataka izrađena je u sustavu Postgres\footnote{https://www.postgresql.org}, koji proširuje jezik SQL i ima reputaciju pouzdanog i moćnog alata za upravljanje bazama podataka, a deployana je na poslužitelju u oblaku Render\footnote{https://render.com}. 
Backend je napisan u Javi te je korišten Spring Boot\footnote{https://spring.io/projects/spring-boot}. Spring Boot je framework temeljen na Java frameworku, dizajniran da autokonfiguracijom i metodom convention-over-configuration pojednostavi izradu aplikacija. 

Za frontend je upotrebljen su standardni jezici HTML, CSS i JavaScript te React\footnote{https://react.dev}, široko upotrebljivana biblioteka za izradu web i native korisničkih sučelja, temeljena na izradni pojedinačnih, jednostavnijih komponenti koje zajedno čine složeni UI, a koju održava Meta. 

Pri izradi, to jest lokalnom testiranju, korišten je Vite\footnote{https://vitejs.dev}. Vite je lokalni server koji prati promjene u datoteci i nakon spremanja automatski osvježava browser kako bi te promjene odmah bile vidljive.

Za ostvarenje interaktivne karte i geokodiranje, korištene su usluge platforme Mapbox\footnote{https://www.mapbox.com}.

\end{document}
