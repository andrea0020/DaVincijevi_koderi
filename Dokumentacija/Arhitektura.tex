\chapter{Arhitektura i dizajn sustava}
		
		\textbf{\textit{dio 1. revizije}}\\

		\textit{ Potrebno je opisati stil arhitekture te identificirati: podsustave, preslikavanje na radnu platformu, spremišta podataka, mrežne protokole, globalni upravljački tok i sklopovsko-programske zahtjeve. Po točkama razraditi i popratiti odgovarajućim skicama:}
	\begin{itemize}
		\item 	\textit{izbor arhitekture temeljem principa oblikovanja pokazanih na predavanjima (objasniti zašto ste baš odabrali takvu arhitekturu)}
		\item 	\textit{organizaciju sustava s najviše razine apstrakcije (npr. klijent-poslužitelj, baza podataka, datotečni sustav, grafičko sučelje)}
		\item 	\textit{organizaciju aplikacije (npr. slojevi frontend i backend, MVC arhitektura) }		
	\end{itemize}
 	Da bismo razradili arhitekturu web aplikacije za poboljšanje dostupnosti knjiga prevedenih na hrvatski i srodne jezike, detaljnije ćemo opisati njene komponente, organizaciju i međusobnu komunikaciju. Evo ključnih elemenata:


        \textbf{1. Stil Arhitekture}\\
        Stil arhitekture bi mogao biti mikroservisni ili slojeviti (npr. MVC - Model-View-Controller). Mikroservisni pristup omogućava modularnost i lako skaliranje, dok slojeviti pristup olakšava razvoj i održavanje. Prilikom razvoja naše aplikacije odabran je MVC model.
        
        \textbf{2. Podsustavi}\\

        Podsustavi se mogu podijeliti na:
        \begin{itemize}
		  \item {Korisničko sučelje (UI): Front-end komponenta zadužena za interakciju s korisnicima.}
		  \item {Poslovna logika: Središnji dio koji upravlja funkcionalnostima aplikacije.}
		  \item {Baza podataka: Spremište podataka gdje se čuvaju informacije o knjigama, korisnicima, ponuditeljima, itd.}		
            \item {Autentikacija i autorizacija: Sustav za upravljanje korisničkim pristupima i ovlastima.}	
             \item {Integracija vanjskih usluga: Za funkcionalnosti poput prikaza karte (npr. integracija OpenStreetMap).}		
	   \end{itemize}

        \textbf{3. Preslikavanje na Radnu Platformu}\\

        Komponente se raspoređuju na servere i klijentske uređaje:
        \begin{itemize}
		  \item {Server: Hostira poslovnu logiku, bazu podataka, autentikaciju i API za integraciju vanjskih usluga.}
		  \item {Klijent: Uređaji korisnika (mobiteli, tableti, računala) s web preglednikom za pristup UI.}
	   \end{itemize}
    
    \textbf{4. Spremišta Podataka}\\

        Centralizirana baza podataka (SQL ili NoSQL) koja pohranjuje:
        \begin{itemize}
		  \item {Podatke o knjigama.}
		  \item {Informacije o korisnicima i ponuditeljima.}
		  \item {Zahtjeve za prijevod i ponude.}				
	   \end{itemize}

    \textbf{5. Mrežni Protokoli}\\
    
    \begin{itemize}
		  \item {HTTP/HTTPS: Za komunikaciju između klijenta i servera.}
		  \item {API protokoli (REST, GraphQL): Za upite i manipulaciju podacima.}
	   \end{itemize}

    \textbf{6. Globalni Upravljački Toke}\\

        Sustav bi koristio zahtjev-odgovor model za komunikaciju između klijenta i servera. Autentikacija i autorizacija korisnika odvijaju se na početku sesije.
    
    \textbf{7. Sklopovsko-Programski Zahtjevi}\\

    \begin{itemize}
		  \item {Server: Pouzdan i skalabilan, s dovoljno resursa za obradu zahtjeva i pohranu podataka.}
		  \item {Klijent: Kompatibilnost s modernim web preglednicima i prilagodljivost različitim veličinama ekrana.}			
	   \end{itemize}

	
		

		

				
		\section{Baza podataka}
			
			\textbf{\textit{dio 1. revizije}}\\
			
		\textit{Potrebno je opisati koju vrstu i implementaciju baze podataka ste odabrali, glavne komponente od kojih se sastoji i slično.}
		
			\subsection{Opis tablica}
			

				\textit{Svaku tablicu je potrebno opisati po zadanom predlošku. Lijevo se nalazi točno ime varijable u bazi podataka, u sredini se nalazi tip podataka, a desno se nalazi opis varijable. Svjetlozelenom bojom označite primarni ključ. Svjetlo plavom označite strani ključ}
				
				
				\begin{longtblr}[
					label=none,
					entry=none
					]{
						width = \textwidth,
						colspec={|X[6,l]|X[6, l]|X[20, l]|}, 
						rowhead = 1,
					} %definicija širine tablice, širine stupaca, poravnanje i broja redaka naslova tablice
					\hline \SetCell[c=3]{c}{\textbf{Knjiga}}	 \\ \hline[3pt]
					\SetCell{LightGreen}ISBN & VARCHAR(13) & \\ \hline
					naslov	& VARCHAR & \\ \hline 
					autori & VARCHAR &   \\ \hline 
					jezik & CHAR(3)	& troslovni kod po ISO 639-1 standardu 	\\ \hline 
					izdavač & VARCHAR & 	\\ \hline
					brojIzdanja & INT & 	\\ \hline 
					godinaIzdanja & INT & 	\\ \hline
					katIzdavača & CHAR(3) & 	\\ \hline
					zahtZaPr & INT & 	\\ \hline
					očuvanost & VARCHAR & 	\\ \hline
					opis & VARCHAR & 	\\ \hline
					slika & BYTEA & 	\\ \hline
					žanr & VARCHAR & 	\\ \hline
					
					\hline \SetCell[c=3]{c}{\textbf{Knjiga_dobavljivost}}	 \\ \hline[3pt]
					\SetCell{LightBlue}ISBN & VARCHAR(13) & \\ \hline
					dobavljivost & CHAR(3) & \\ \hline 
					
					\hline \SetCell[c=3]{c}{\textbf{Ponuditelj}}	 \\ \hline[3pt]
					\SetCell{LightGreen}korisničkoIme & VARCHAR(32) & \\ \hline
					lozinka	& VARCHAR(32) &   	\\ \hline 
					naziv & VARCHAR &   \\ \hline 
					adresa & VARCHAR	&  		\\ \hline 
					epošta & VARCHAR & 	\\ \hline
					telefon & VARCHAR & 	\\ \hline 
					godinaIzdanja & INT & 	\\ \hline
					katIzdavača & CHAR(3) & 	\\ \hline
					zahtZaPr & INT & 	\\ \hline
					lokacija & POINT & 	\\ \hline
					tip & VARCHAR & izdavač, antikvarijat, preprodavač	\\ \hline
					
					\hline \SetCell[c=3]{c}{\textbf{čekaOdobrenje}}	 \\ \hline[3pt]
					\SetCell{LightBlue}korisničkoIme & VARCHAR(32) & \\ \hline
					status & VARCHAR & \\ \hline 
					
					\hline \SetCell[c=3]{c}{\textbf{nudi}}	 \\ \hline[3pt]
					\SetCell{LightBlue}naziv & VARCHAR & \\ \hline
					\SetCell{LightBlue}ISBN & CHAR(13) &   	\\ \hline 
					broj & INT &   \\ \hline 
					cijena & DECIMAL(6, 2) & \\ \hline 
				\end{longtblr}
				
				
			
			\subsection{Dijagram baze podataka}
				\textit{ U ovom potpoglavlju potrebno je umetnuti dijagram baze podataka. Primarni i strani ključevi moraju biti označeni, a tablice povezane. Bazu podataka je potrebno normalizirati. Podsjetite se kolegija "Baze podataka".}
			
			\eject
			
			
		\section{Dijagram razreda}
		
			\textit{Potrebno je priložiti dijagram razreda s pripadajućim opisom. Zbog preglednosti je moguće dijagram razlomiti na više njih, ali moraju biti grupirani prema sličnim razinama apstrakcije i srodnim funkcionalnostima.}\\
			
			\textbf{\textit{dio 1. revizije}}\\
			
			\textit{Prilikom prve predaje projekta, potrebno je priložiti potpuno razrađen dijagram razreda vezan uz \textbf{generičku funkcionalnost} sustava. Ostale funkcionalnosti trebaju biti idejno razrađene u dijagramu sa sljedećim komponentama: nazivi razreda, nazivi metoda i vrste pristupa metodama (npr. javni, zaštićeni), nazivi atributa razreda, veze i odnosi između razreda.}\\
			
			\textbf{\textit{dio 2. revizije}}\\			
			
			\textit{Prilikom druge predaje projekta dijagram razreda i opisi moraju odgovarati stvarnom stanju implementacije}
			
			
			
			\eject
		
		\section{Dijagram stanja}
			
			
			\textbf{\textit{dio 2. revizije}}\\
			
			\textit{Potrebno je priložiti dijagram stanja i opisati ga. Dovoljan je jedan dijagram stanja koji prikazuje \textbf{značajan dio funkcionalnosti} sustava. Na primjer, stanja korisničkog sučelja i tijek korištenja neke ključne funkcionalnosti jesu značajan dio sustava, a registracija i prijava nisu. }
			
			
			\eject 
		
		\section{Dijagram aktivnosti}
			
			\textbf{\textit{dio 2. revizije}}\\
			
			 \textit{Potrebno je priložiti dijagram aktivnosti s pripadajućim opisom. Dijagram aktivnosti treba prikazivati značajan dio sustava.}
			
			\eject
		\section{Dijagram komponenti}
		
			\textbf{\textit{dio 2. revizije}}\\
		
			 \textit{Potrebno je priložiti dijagram komponenti s pripadajućim opisom. Dijagram komponenti treba prikazivati strukturu cijele aplikacije.}
